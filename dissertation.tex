% Dissertation: Identifying Depression Through Speech
% University of Glasgow - Level 4 Computer Science
% Author: Nile
% 
\documentclass{l4proj}

% Additional packages
\usepackage{pdfpages}
\usepackage{booktabs}
\usepackage{multirow}
\usepackage{graphicx}
\usepackage{amsmath}
\usepackage{hyperref}

% Set graphics path
\graphicspath{{figures/}{figures/advanced/}}

\begin{document}

%==============================================================================
% METADATA
%==============================================================================
\title{Identifying Depression Through Speech: A Comparative Analysis of Acoustic Features in Read and Spontaneous Speech}
\author{Nile}
\date{\today}

\maketitle

%==============================================================================
% ABSTRACT
%==============================================================================
\begin{abstract}
Depression is a leading cause of disability worldwide, yet diagnosis relies heavily on subjective self-report and clinical observation. Speech-based biomarkers offer a promising avenue for objective, non-invasive assessment. While recent research has achieved high classification accuracy using deep learning approaches, the resulting models often lack interpretability, limiting their clinical utility.

This project investigated which acoustic features of speech are most predictive of depression, with particular focus on comparing read versus spontaneous speech modalities. Using the ANDROIDS corpus (228 recordings from 118 speakers with clinical diagnoses), 88 acoustic features from the eGeMAPS standard set were extracted and analysed using Support Vector Machine and Random Forest classifiers.

Results demonstrated that spontaneous interview speech (87.1\% accuracy) significantly outperformed reading tasks (72.3\% accuracy, p = 0.0055) for depression detection. Critically, the most predictive features differed between tasks: reading speech was characterised by spectral slope and MFCC features, while interview speech was dominated by variability measures—particularly spectral flux variability, voice quality dynamics, and pause characteristics.

These findings suggest that the ``flat'' or ``monotonous'' quality of depressed speech reflects reduced dynamic modulation rather than altered mean acoustic properties. The prominence of pause-related features for spontaneous speech aligns with known psychomotor retardation in depression. This work demonstrates that interpretable traditional methods can achieve competitive performance while providing clinically meaningful insights into how depression manifests in speech.
\end{abstract}

%==============================================================================
% ACKNOWLEDGEMENTS
%==============================================================================
\chapter*{Acknowledgements}
% Add acknowledgements here
%
% Thank you to my supervisor for their guidance throughout this project.
% Thanks also to the creators of the ANDROIDS corpus for making their data publicly available.
%

%==============================================================================
% EDUCATIONAL CONSENT (optional - uncomment to sign)
%==============================================================================
%\def\consentname {Nile} % your full name
%\def\consentdate {\today} % the date you agree
%
\educationalconsent

%==============================================================================
\tableofcontents

%==============================================================================
% CHAPTERS
%==============================================================================

% Note: Each chapter file begins with \chapter{...}
% The \pagenumbering{arabic} command is added after the first \chapter

\chapter{Introduction}

\section{Motivation}

Depression affects over 280 million people globally, making it one of the leading causes of disability worldwide~\cite{WHO2024}. In the United Kingdom alone, approximately one in six adults experiences depression or anxiety in any given week, with an estimated economic burden of £105 billion annually~\cite{McManus2016, OECD2018}.

Despite its prevalence, depression remains significantly underdiagnosed and undertreated. Current diagnostic methods rely primarily on clinical interviews and self-report questionnaires such as the Patient Health Questionnaire (PHQ-9)~\cite{Kroenke2001}. These approaches face several critical limitations:

\begin{itemize}
    \item \textbf{Subjectivity:} Diagnostic agreement between clinicians can be inconsistent, with inter-rater reliability as low as 0.28 kappa for certain depression subtypes~\cite{Regier2013}.
    
    \item \textbf{Self-report bias:} Patients may underreport symptoms due to stigma, lack of insight, or difficulty articulating their experiences.
    
    \item \textbf{Resource constraints:} Clinical assessments require trained mental health professionals, limiting scalability and accessibility.
    
    \item \textbf{Access barriers:} Many individuals, particularly in underserved communities, lack access to specialist care.
\end{itemize}

These limitations have motivated research into objective, scalable biomarkers for depression. Among the most promising candidates is \textit{speech}. Depression affects cognitive processes that directly influence speech production—changes that are often involuntary and difficult to consciously mask~\cite{Cummins2015}. Recording and analysing speech is non-invasive, inexpensive, and can be performed remotely, making it an attractive option for large-scale screening.

Recent advances in machine learning have enabled automatic detection of depression from speech with accuracy comparable to general practitioners~\cite{Tao2024}. However, most research has focused on maximising classification accuracy, treating models as ``black boxes.'' This raises a critical question: \textit{which aspects of speech actually indicate depression?}

Understanding which acoustic features predict depression is essential for:
\begin{enumerate}
    \item \textbf{Clinical utility:} Clinicians cannot act on ``83\% accuracy''—they need interpretable markers.
    \item \textbf{Scientific understanding:} Identifying robust biomarkers advances our understanding of how depression manifests in behaviour.
    \item \textbf{System design:} Knowing which features matter informs what data to collect and how to deploy screening tools.
\end{enumerate}

This dissertation addresses this gap by systematically analysing which acoustic features are most predictive of depression, and how their importance differs between controlled (read) and naturalistic (spontaneous) speech.


\section{Aims}

This project aims to:

\begin{enumerate}
    \item \textbf{Identify predictive features:} Determine which acoustic and prosodic features are most strongly associated with depression, using interpretable machine learning techniques.
    
    \item \textbf{Compare speech tasks:} Analyse whether the same features predict depression in read speech (controlled phonetic content) versus spontaneous speech (natural conversation), and which task yields more reliable detection.
    
    \item \textbf{Provide interpretable analysis:} Move beyond accuracy metrics to explain \textit{why} models predict depression, using feature importance methods such as SHAP values.
\end{enumerate}

The research question guiding this work is:

\begin{quote}
\textit{Which acoustic features of speech are most predictive of depression, and how do they differ between read and spontaneous speech?}
\end{quote}


\section{Outline}

This dissertation is structured as follows:

\begin{description}
    \item[Chapter 2: Background] Reviews the relationship between depression and speech production, formally defines acoustic features used in analysis, surveys machine learning approaches to depression detection, and discusses related work including the ANDROIDS corpus.
    
    \item[Chapter 3: Design] Describes the experimental methodology, including data selection, feature extraction pipeline, classification approach, and feature importance analysis techniques.
    
    \item[Chapter 4: Implementation] Details the technical implementation, including tools used, data processing pipeline, and experimental setup.
    
    \item[Chapter 5: Evaluation] Presents classification results, feature importance rankings, and comparison between read and spontaneous speech tasks.
    
    \item[Chapter 6: Discussion] Critically analyses findings, discusses limitations, compares results with existing literature, and considers clinical implications.
    
    \item[Chapter 7: Conclusion] Summarises contributions and suggests directions for future work.
\end{description}

% Add page numbering reset after first chapter heading
% (This is handled in 01_introduction.tex)

\chapter{Background}

This chapter establishes the theoretical and technical foundations necessary for understanding the present research. It begins by examining the clinical context of depression and its effects on speech production, then explores how these speech changes can be measured through acoustic features, before reviewing existing computational approaches and identifying the gap this work addresses.

\section{Depression and Its Effects on Speech}

\subsection{The Clinical Context}

Major Depressive Disorder (MDD) is characterised by persistent feelings of sadness, hopelessness, and diminished interest in activities. Beyond its emotional symptoms, depression profoundly affects cognitive and motor functioning---effects that manifest in observable changes to speech production.

The neurobiological basis for these speech changes is well-established. Depression is associated with dysfunction in prefrontal cortical regions responsible for executive control, as well as disruption to dopaminergic pathways that regulate motor initiation and reward processing \cite{Sobin1997}. Psychomotor retardation---the slowing of physical and cognitive processes---is a core feature of melancholic depression, affecting approximately 70\% of hospitalised patients. This retardation directly impacts the motor planning and execution required for fluent speech production.

At the cognitive level, depression impairs working memory capacity and attentional control \cite{Joormann2011}. Speech production is a demanding cognitive task: Levelt's influential model identifies multiple processing stages including conceptualisation, formulation, and articulation, all operating in parallel and requiring continuous monitoring \cite{Levelt1989}. When cognitive resources are depleted by rumination or slowed by psychomotor retardation, these processes become less efficient, manifesting as increased hesitations, longer pauses, and reduced fluency.

\subsection{How Depression Manifests in Speech}

The observable speech changes in depression can be understood through three interconnected mechanisms.

\paragraph{Prosodic flattening} reflects the emotional blunting characteristic of depression. Healthy emotional expression involves dynamic modulation of pitch, loudness, and timing---what linguists term prosody. Depressed individuals show reduced emotional reactivity \cite{Bylsma2008}, which manifests as diminished prosodic variation: narrower pitch range, reduced loudness dynamics, and more monotonous delivery. Importantly, this flattening appears to be involuntary; speakers cannot easily mask it, making it a potentially robust biomarker.

\paragraph{Temporal disruption} arises from psychomotor retardation and cognitive load. The production of spontaneous speech requires real-time lexical retrieval, syntactic planning, and articulatory coordination. When these processes are slowed, the result is increased pause duration, lower speech rate, and more frequent hesitations. Cannizzaro et al. found that pause-related features were among the strongest acoustic correlates of depression severity \cite{Cannizzaro2004}.

\paragraph{Voice quality changes} reflect alterations to laryngeal function. Depression is associated with increased muscle tension and changes to respiratory control, which affect vocal fold vibration patterns. These changes manifest as increased jitter (cycle-to-cycle pitch variation), shimmer (amplitude variation), and breathiness. Ozdas et al. demonstrated that voice quality measures could distinguish depressed from non-depressed speakers with reasonable accuracy, though results have been less consistent across studies than prosodic measures \cite{Ozdas2004}.

A critical observation from the literature is that \textbf{variability measures often outperform means}. It is not simply that depressed speakers have lower average pitch or speak more slowly; rather, they show reduced \textit{dynamic range}---less variation around whatever their baseline might be. This insight, while present in the clinical literature, has been underexplored in computational approaches that often focus on mean values.

\section{Acoustic Feature Extraction}

\subsection{From Speech Signal to Measurable Features}

To analyse speech computationally, the raw audio signal must be transformed into a structured representation. This process involves extracting acoustic features---numerical descriptors that capture relevant properties of the speech signal. The choice of features is consequential: different features capture different aspects of speech, and their relevance to depression detection varies.

\subsection{Prosodic Features}

\textbf{Fundamental frequency (F0)}, corresponding to vocal fold vibration rate and perceived as pitch, is the most studied feature in depression research. F0 is typically extracted using autocorrelation or cepstral methods, producing a time series that can be summarised through statistical functionals: mean, standard deviation, percentiles, and slope measures. The theoretical motivation is clear: F0 variation reflects emotional expressiveness, and reduced variation is a hallmark of depressive flattening.

However, F0-based features have limitations. Extraction algorithms can be unreliable in noisy conditions or with certain voice types, and cross-speaker normalisation is challenging due to large individual differences in baseline pitch. Some researchers argue that F0 variability, while theoretically motivated, has shown inconsistent results precisely because of these measurement challenges \cite{Cummins2015}.

\textbf{Loudness and energy features} capture vocal intensity and its dynamics. Root mean square energy provides an overall measure, while loudness slope features (both rising and falling) characterise the dynamic envelope of speech. These features are more robust to extraction than F0 but potentially confounded by recording conditions and speaker-microphone distance.

\subsection{Spectral Features}

\textbf{Mel-Frequency Cepstral Coefficients (MFCCs)} are the standard spectral representation in speech processing. The computation involves applying a mel-scaled filterbank to the power spectrum, taking the logarithm, and applying a discrete cosine transform. The resulting coefficients approximate human auditory perception and capture vocal tract configuration.

MFCCs are ubiquitous in speech technology but their relevance to depression detection deserves scrutiny. They were developed for speech recognition, where the goal is capturing phonetic content. For depression detection, their utility is empirical rather than theoretically motivated---they work reasonably well, but it is unclear which aspects of the spectral shape they capture are actually relevant to depressive speech changes. This ambiguity makes interpretation difficult.

\textbf{Spectral flux} measures frame-to-frame spectral change, capturing the dynamic modulation of the voice. Unlike MFCCs, which describe static spectral shape, spectral flux characterises how that shape evolves over time. This may be more directly relevant to the reduced dynamic variation observed in depression.

\textbf{Voice quality features} including the Hammarberg index (ratio of energy below vs above 2kHz) and alpha ratio (similar spectral balance measure) capture characteristics like breathiness and vocal strain. These have clear physiological interpretations related to laryngeal function.

\subsection{The eGeMAPS Feature Set}

Recognising the proliferation of ad-hoc feature sets in affective computing, Eyben et al. proposed the extended Geneva Minimalistic Acoustic Parameter Set (eGeMAPS): a standardised set of 88 features designed specifically for affective computing and clinical speech analysis \cite{Eyben2016}.

The eGeMAPS set was motivated by several considerations. First, \textbf{parsimony}: rather than thousands of features requiring aggressive dimensionality reduction, 88 features provide comprehensive coverage while remaining manageable. Second, \textbf{interpretability}: each feature has a defined acoustic meaning, unlike the opaque representations learned by neural networks. Third, \textbf{standardisation}: using a common feature set enables comparison across studies.

The 88 features span frequency (F0 statistics and dynamics), energy (loudness and its variations), spectral (MFCCs 1-4, spectral flux, formants), and temporal (voiced/unvoiced segment statistics) domains. Functionals---statistical summaries computed over the entire recording---transform frame-level measurements into fixed-length vectors.

This work adopts eGeMAPS for several reasons aligned with the research question. The focus on feature interpretability requires features with clear meanings; eGeMAPS provides this. The goal of identifying \textit{which} features matter requires a manageable set that can be analysed individually; eGeMAPS is sized appropriately. And the use of a standard set enables comparison with prior work on the same corpus.

\section{Computational Approaches to Depression Detection}

\subsection{Traditional Machine Learning}

Early computational work on depression detection employed classical machine learning algorithms with handcrafted features. Support Vector Machines (SVMs) were particularly popular, performing well with high-dimensional feature vectors and limited training data. Random Forests offered the additional advantage of built-in feature importance estimates through Gini impurity or permutation-based measures.

These approaches achieved reasonable accuracy---typically 65-80\% on binary classification tasks---while maintaining interpretability. A trained SVM or Random Forest can be interrogated: which features received high weights? Which drove particular predictions? This transparency is valuable for clinical applications where explanations matter.

\subsection{Deep Learning Approaches}

The deep learning revolution reached depression detection around 2016. Convolutional Neural Networks (CNNs) applied to spectrograms, Recurrent Neural Networks (RNNs) modelling temporal dynamics, and attention mechanisms focusing on relevant speech segments all showed improvements over traditional methods.

Ma et al. demonstrated that CNNs could learn depression-relevant representations directly from spectrograms, bypassing handcrafted features entirely \cite{Ma2016}. Subsequent work, including Tao et al.'s Multi-Local Attention approach, achieved state-of-the-art results on benchmark datasets \cite{Tao2023MLA}.

However, these improvements came at a cost. Neural networks are notoriously difficult to interpret. While post-hoc explanation methods like SHAP and attention visualisation offer some insight, they do not provide the direct feature-level understanding that traditional methods afford. For a researcher asking ``which acoustic properties of speech are affected by depression?'', a neural network's answer is unsatisfyingly opaque.

This trade-off between accuracy and interpretability is central to the present work. The question is not simply ``can we detect depression?'' but ``what can detection tell us about how depression affects speech?''

\section{Related Work and Datasets}

\subsection{The AVEC Challenge Series}

The Audio/Visual Emotion Challenge (AVEC) series established benchmarks for affective computing, including depression detection. AVEC 2016, 2017, and 2019 featured depression sub-challenges using the DAIC-WOZ corpus \cite{Gratch2014}.

Examining the winning approaches reveals important patterns. The 2016 winner combined audio, video, and text modalities, achieving 4.99 MAE on PHQ-8 prediction. Subsequent challenges showed similar trends: multimodal approaches consistently outperformed unimodal ones, often by substantial margins. This raises questions about speech-only approaches: is acoustic information sufficient, or are visual and linguistic cues essential?

The improvement from multimodality is not uniform across features. Analysis of challenge results suggests that acoustic features contribute most to detecting severe depression, while linguistic features (word choice, response patterns) may better capture mild cases. This observation has implications for system design but has received limited attention in the literature.

\subsection{The DAIC-WOZ Corpus}

The Distress Analysis Interview Corpus (DAIC-WOZ) has been the primary benchmark for depression detection research \cite{Gratch2014}. It contains clinical interviews with participants who completed PHQ-8 self-report questionnaires.

Despite its influence, DAIC-WOZ has limitations relevant to this work. First, labels are self-reported questionnaire scores, not clinical diagnoses---introducing potential noise from response biases. Second, all speech is spontaneous interview responses; there is no controlled reading task for comparison. Third, access is restricted, limiting reproducibility.

\subsection{The ANDROIDS Corpus}

Tao et al. introduced the ANDROIDS corpus specifically to address limitations in existing datasets \cite{Tao2023ANDROIDS}. Three features make it particularly suitable for the present research.

\textbf{Clinical labels}: Unlike DAIC-WOZ's self-report scores, ANDROIDS participants were labelled based on psychiatric assessment by clinicians. This provides more reliable ground truth, though the binary healthy/depressed distinction loses severity information.

\textbf{Dual speech tasks}: Critically, ANDROIDS includes both reading and spontaneous interview tasks from the same participants. This enables direct comparison of how depression manifests across speech modalities---a comparison impossible with DAIC-WOZ. Reading provides controlled linguistic content, isolating acoustic characteristics; interviews capture naturalistic speech with all its cognitive demands.

\textbf{Public availability}: ANDROIDS is freely accessible for research, enabling reproducibility.

Tao et al.'s own analysis of ANDROIDS achieved 83.4\% accuracy on reading and 81.6\% on spontaneous speech using deep learning \cite{Tao2024}. However, their focus was on detection accuracy, not feature analysis. They did not systematically investigate which acoustic features drove their model's predictions or whether the same features mattered for both tasks.

\subsection{The Gap: Feature Interpretability}

Surveying the literature reveals a consistent pattern: most work prioritises detection accuracy over understanding. Researchers report overall accuracy, F1 scores, and comparisons to baselines, but rarely analyse \textit{which features} contribute most to predictions or \textit{why}.

This gap exists for understandable reasons. The field has been driven by challenge benchmarks that reward accuracy, not interpretability. Deep learning methods that dominate leaderboards are inherently opaque. And rigorous feature importance analysis requires careful methodology that adds complexity beyond simply training a classifier.

Yet the gap has costs. Clinicians cannot trust systems they do not understand. The scientific goal of understanding depression's effects on speech is not served by black-box predictions. And practical systems cannot be optimised without knowing which features matter.

The ANDROIDS corpus, with its dual tasks and clinical labels, provides an ideal testbed for addressing this gap. By analysing feature importance across reading and spontaneous speech, we can ask: which acoustic features are most predictive of depression? Do the same features matter in controlled versus naturalistic speech? What does this reveal about how depression affects speech production?

\section{Summary}

Depression affects speech through multiple mechanisms: psychomotor retardation disrupts timing, emotional blunting reduces prosodic variation, and altered laryngeal function changes voice quality. These effects can be captured through acoustic features spanning prosodic, spectral, and temporal domains.

Computational approaches have achieved reasonable detection accuracy, with deep learning methods currently dominant. However, the field has prioritised accuracy over interpretability, leaving important questions unanswered about which features drive predictions and why.

The ANDROIDS corpus offers unique advantages for addressing these questions: clinical labels, dual speech tasks, and public availability. This work uses ANDROIDS with interpretable machine learning methods to analyse feature importance, directly addressing the gap in current research.


\chapter{Methodology}

This chapter details the experimental methodology employed in this study. It covers the research approach, data selection, feature extraction process, classification methods, and evaluation strategy.

\section{Research Approach}

This study adopts an empirical, quantitative approach to investigate which acoustic features of speech are most predictive of depression, with a particular focus on comparing read versus spontaneous speech modalities. The experimental design follows a supervised machine learning paradigm:

\begin{enumerate}
    \item Extract standardised acoustic features from speech recordings
    \item Train classification models to distinguish depressed from healthy individuals
    \item Analyse feature importance to identify the most predictive acoustic markers
    \item Compare results across speech tasks (reading versus interview)
\end{enumerate}

This approach was chosen over deep learning methods specifically because the research question centres on \textit{interpretability}—understanding which features contribute to depression detection—rather than maximising classification accuracy alone. While neural network approaches have achieved higher accuracy in some studies \cite{cummins2015review}, their ``black box'' nature makes clinical interpretation difficult.

\section{Dataset Selection}

\subsection{The ANDROIDS Corpus}

The ANDROIDS (ANDRoid corpus fOr Identification of Depression and Suicide risk) corpus \cite{dinkel2019androids} was selected for this study. It contains speech recordings from 118 Italian speakers, comprising both individuals with clinical depression diagnoses and healthy controls.

\subsubsection{Corpus Characteristics}

\begin{table}[h]
\centering
\caption{ANDROIDS corpus composition}
\label{tab:corpus-composition}
\begin{tabular}{lcccc}
\hline
\textbf{Group} & \textbf{Reading} & \textbf{Interview} & \textbf{Total} \\
\hline
Healthy Controls (HC) & 54 & 52 & 106 \\
Patients (PT) & 58 & 64 & 122 \\
\textbf{Total} & 112 & 116 & 228 \\
\hline
\end{tabular}
\end{table}

Key advantages of this corpus for the current study:

\begin{itemize}
    \item \textbf{Dual speech tasks:} Includes both reading and interview recordings from the same participants, enabling direct comparison of speech modalities
    \item \textbf{Clinical diagnoses:} Depression labels are based on psychiatric assessment, not self-report questionnaires
    \item \textbf{Public availability:} Enables reproducibility of results
    \item \textbf{Controlled recording conditions:} Minimises acoustic variation from environmental factors
\end{itemize}

\subsubsection{Speech Tasks}

The corpus contains two distinct speech tasks:

\paragraph{Reading Task:} Participants read a standardised Italian text passage aloud. This provides controlled linguistic content, allowing acoustic analysis independent of spontaneous language production processes.

\paragraph{Interview Task:} Participants engaged in semi-structured interviews covering topics such as daily routines, emotional experiences, and future plans. This elicits naturalistic speech with variable linguistic content.

The inclusion of both tasks is central to the research question, as the cognitive demands of reading versus spontaneous speech differ substantially, and these may interact differently with depression-related speech patterns.

\subsection{Ethical Considerations}

The ANDROIDS corpus is publicly available for research purposes. All recordings were collected with informed consent, and data are anonymised. No additional ethics approval was required for this secondary analysis of existing data.

\section{Feature Extraction}

\subsection{The eGeMAPS Feature Set}

Acoustic features were extracted using the extended Geneva Minimalistic Acoustic Parameter Set (eGeMAPS) \cite{eyben2016geneva}, a standardised feature set designed specifically for affective computing and clinical speech analysis. The eGeMAPS set was chosen for several reasons:

\begin{enumerate}
    \item \textbf{Standardisation:} Enables comparison with published literature using the same features
    \item \textbf{Interpretability:} Features have clear acoustic and physiological interpretations
    \item \textbf{Comprehensiveness:} Covers prosody, voice quality, spectral characteristics, and temporal dynamics
    \item \textbf{Manageable dimensionality:} 88 features provide rich representation without excessive complexity
\end{enumerate}

\subsection{Feature Categories}

The 88 eGeMAPS features span several acoustic domains:

\begin{table}[h]
\centering
\caption{eGeMAPS feature categories}
\label{tab:feature-categories}
\begin{tabular}{lp{7cm}}
\hline
\textbf{Category} & \textbf{Features} \\
\hline
Frequency (F0) & Pitch mean, standard deviation, percentiles, slopes \\
Energy/Loudness & Mean loudness, peak rate, slopes, percentiles \\
Spectral & MFCCs (1-4), spectral flux, formant frequencies, bandwidth \\
Voice Quality & Jitter, shimmer, Hammarberg index, alpha ratio \\
Temporal & Voiced/unvoiced segment duration, speech rate \\
\hline
\end{tabular}
\end{table}

\subsection{Extraction Process}

Feature extraction was performed using the openSMILE toolkit \cite{eyben2010opensmile} via its Python bindings. For each audio file, the extraction pipeline:

\begin{enumerate}
    \item Loaded the WAV file (16-bit PCM format)
    \item Applied the eGeMAPS configuration with functional statistics
    \item Generated a single 88-dimensional feature vector per recording
    \item Stored features alongside metadata (speaker ID, condition, task)
\end{enumerate}

The ``functionals'' level was used, computing statistical summaries (mean, standard deviation, percentiles) over frame-level features across the entire recording. This produces a fixed-length representation regardless of recording duration.

\section{Classification Methods}

Two classification algorithms were employed: Support Vector Machines (SVM) and Random Forest. Both are well-established methods for speech-based depression detection with complementary strengths.

\subsection{Support Vector Machine}

SVMs find the hyperplane that maximally separates classes in feature space. A radial basis function (RBF) kernel was used to capture nonlinear relationships:

\begin{equation}
    K(x_i, x_j) = \exp\left(-\gamma \|x_i - x_j\|^2\right)
\end{equation}

Feature standardisation (zero mean, unit variance) was applied prior to SVM training, as the RBF kernel is sensitive to feature scales.

\textbf{Hyperparameters:}
\begin{itemize}
    \item Kernel: RBF
    \item Regularisation (C): 1.0
    \item Gamma: Scale (1 / (n\_features $\times$ variance))
\end{itemize}

\subsection{Random Forest}

Random Forest constructs an ensemble of decision trees, each trained on a bootstrap sample with random feature subsets. The ensemble averages predictions across trees, reducing overfitting and providing built-in feature importance estimates.

\textbf{Hyperparameters:}
\begin{itemize}
    \item Number of trees: 100
    \item Maximum depth: 10
    \item Split criterion: Gini impurity
\end{itemize}

The depth limit was imposed to prevent overfitting on the relatively small dataset.

\section{Evaluation Strategy}

\subsection{Cross-Validation}

All results are reported using 5-fold stratified cross-validation. Stratification ensures each fold maintains the same class distribution as the full dataset, important given the mild class imbalance.

For each fold:
\begin{enumerate}
    \item Train on 80\% of data
    \item Test on held-out 20\%
    \item Record accuracy and F1 score
\end{enumerate}

Final metrics are the mean and standard deviation across folds.

\subsection{Evaluation Metrics}

\paragraph{Accuracy:} Proportion of correctly classified samples:
\begin{equation}
    \text{Accuracy} = \frac{TP + TN}{TP + TN + FP + FN}
\end{equation}

\paragraph{F1 Score:} Harmonic mean of precision and recall, providing balanced evaluation under class imbalance:
\begin{equation}
    F1 = 2 \cdot \frac{\text{Precision} \cdot \text{Recall}}{\text{Precision} + \text{Recall}}
\end{equation}

\subsection{Feature Importance Analysis}

To address the research question of which features are most predictive, two complementary importance measures were computed:

\paragraph{Gini Importance:} Measures the total decrease in node impurity (Gini index) attributable to each feature across all trees in the Random Forest. While computationally efficient, Gini importance can be biased toward high-cardinality features.

\paragraph{Permutation Importance:} Measures the decrease in model accuracy when feature values are randomly shuffled. This provides a more reliable estimate of true predictive value:

\begin{equation}
    I_{\text{perm}}(f) = \frac{1}{K} \sum_{k=1}^{K} \left[ \text{Acc}_{\text{original}} - \text{Acc}_{\text{shuffled}}^{(k)} \right]
\end{equation}

where $K=10$ permutations were used per feature.

\section{Reproducibility}

All code, feature files, and analysis scripts are available in the project repository. The complete analysis pipeline can be reproduced by:

\begin{enumerate}
    \item Obtaining the ANDROIDS corpus from the original authors
    \item Running the feature extraction script (\texttt{extract\_features.py})
    \item Running the analysis script (\texttt{run\_analysis.py})
\end{enumerate}

Random seeds were fixed for all stochastic processes to ensure reproducibility.

\section{Summary}

This methodology enables systematic investigation of the research question through:

\begin{itemize}
    \item A carefully selected corpus with both speech modalities
    \item Standardised, interpretable acoustic features (eGeMAPS)
    \item Robust classification with established algorithms
    \item Feature importance analysis for clinical interpretability
    \item Rigorous cross-validation for reliable performance estimates
\end{itemize}

The following chapter details the implementation of this methodology, including technical specifications and code organisation.


\chapter{Implementation}

This chapter describes the technical implementation of the methodology outlined in Chapter 3. It covers the development environment, data processing pipeline, model training procedures, and analysis workflow.

\section{Development Environment}

\subsection{Hardware}

All experiments were conducted on a personal computer with the following specifications:

\begin{itemize}
    \item \textbf{Processor:} Apple M1 / Intel Core (macOS system)
    \item \textbf{Memory:} 8-16 GB RAM
    \item \textbf{Storage:} SSD (sufficient for 4GB corpus + working files)
\end{itemize}

The computational requirements of this project are modest—traditional machine learning methods and feature extraction do not require GPU acceleration.

\subsection{Software Stack}

\begin{table}[h]
\centering
\caption{Software dependencies}
\label{tab:software}
\begin{tabular}{lll}
\hline
\textbf{Component} & \textbf{Version} & \textbf{Purpose} \\
\hline
Python & 3.11+ & Programming language \\
openSMILE & 2.5.0 & Acoustic feature extraction \\
scikit-learn & 1.3+ & Machine learning algorithms \\
pandas & 2.0+ & Data manipulation \\
NumPy & 1.24+ & Numerical computing \\
matplotlib & 3.7+ & Visualisation \\
seaborn & 0.12+ & Statistical plots \\
tqdm & 4.65+ & Progress bars \\
\hline
\end{tabular}
\end{table}

A virtual environment (\texttt{venv}) was used to isolate dependencies and ensure reproducibility.

\section{Data Processing Pipeline}

\subsection{Corpus Organisation}

The ANDROIDS corpus was organised hierarchically:

\begin{verbatim}
data/Androids-Corpus/
├── Reading-Task/
│   └── audio/
│       ├── HC/     # Healthy controls (54 files)
│       └── PT/     # Patients (58 files)
└── Interview-Task/
    └── audio/
        ├── HC/     # Healthy controls (52 files)
        └── PT/     # Patients (64 files)
\end{verbatim}

Each audio file follows the naming convention: \texttt{nn\_XGmm\_t.wav}, where:
\begin{itemize}
    \item \texttt{nn} — Speaker ID (two digits)
    \item \texttt{X} — Condition (P = patient, C = control)
    \item \texttt{G} — Gender (M = male, F = female)
    \item \texttt{mm} — Age (two digits)
    \item \texttt{t} — Education level
\end{itemize}

\subsection{Feature Extraction Implementation}

The feature extraction script (\texttt{extract\_features.py}) processes all audio files and generates feature matrices:

\begin{verbatim}
# Initialize OpenSMILE with eGeMAPS
smile = opensmile.Smile(
    feature_set=opensmile.FeatureSet.eGeMAPSv02,
    feature_level=opensmile.FeatureLevel.Functionals,
)

# Extract features from each file
for wav_path in audio_files:
    features = smile.process_file(str(wav_path))
    # Parse metadata from filename
    # Append to feature matrix
\end{verbatim}

The script:
\begin{enumerate}
    \item Iterates through all WAV files in the corpus
    \item Extracts 88 eGeMAPS features per recording
    \item Parses metadata from filenames (speaker ID, condition, demographics)
    \item Combines features and metadata into a single DataFrame
    \item Saves results in CSV and pickle formats
\end{enumerate}

\subsubsection{Output Files}

\begin{itemize}
    \item \texttt{all\_features.csv} — Complete feature matrix (228 samples × 97 columns)
    \item \texttt{reading\_features.csv} — Reading task subset (112 samples)
    \item \texttt{interview\_features.csv} — Interview task subset (116 samples)
    \item \texttt{all\_features.pkl} — Binary format for faster loading
\end{itemize}

\section{Model Training Implementation}

\subsection{Data Preparation}

Before training, features are separated from metadata:

\begin{verbatim}
metadata_cols = ['filename', 'speaker_id', 'condition',
                 'gender', 'age', 'education', 'label',
                 'task', 'depression']
feature_cols = [c for c in df.columns if c not in metadata_cols]

X = task_df[feature_cols].values  # Shape: (n_samples, 88)
y = task_df['depression'].values  # Shape: (n_samples,)
\end{verbatim}

\subsection{Cross-Validation Setup}

Stratified 5-fold cross-validation ensures class balance in each split:

\begin{verbatim}
cv = StratifiedKFold(n_splits=5, shuffle=True, random_state=42)
\end{verbatim}

The fixed random seed ensures reproducibility—the same folds are generated on each run.

\subsection{SVM Training}

The SVM classifier uses a scikit-learn Pipeline to combine feature scaling with classification:

\begin{verbatim}
svm_pipe = Pipeline([
    ('scaler', StandardScaler()),
    ('svm', SVC(kernel='rbf', C=1.0, random_state=42))
])

svm_scores = cross_val_score(svm_pipe, X, y, cv=cv,
                              scoring='accuracy')
\end{verbatim}

StandardScaler transforms features to zero mean and unit variance—essential for RBF kernel SVMs, which are sensitive to feature magnitudes.

\subsection{Random Forest Training}

Random Forest does not require feature scaling:

\begin{verbatim}
rf = RandomForestClassifier(
    n_estimators=100,
    random_state=42,
    n_jobs=-1,
    max_depth=10
)

rf_scores = cross_val_score(rf, X, y, cv=cv,
                            scoring='accuracy')
\end{verbatim}

Key parameters:
\begin{itemize}
    \item \texttt{n\_estimators=100} — Sufficient trees for stable importance estimates
    \item \texttt{max\_depth=10} — Prevents overfitting on small dataset
    \item \texttt{n\_jobs=-1} — Utilises all CPU cores for parallel training
\end{itemize}

\section{Feature Importance Implementation}

\subsection{Gini Importance}

After training on the full dataset, Random Forest provides built-in feature importance:

\begin{verbatim}
rf.fit(X, y)

rf_importance = pd.DataFrame({
    'feature': feature_cols,
    'importance': rf.feature_importances_
}).sort_values('importance', ascending=False)
\end{verbatim}

Gini importance measures how much each feature contributes to reducing impurity across all decision tree splits.

\subsection{Permutation Importance}

For more robust importance estimates, permutation importance shuffles each feature and measures accuracy degradation:

\begin{verbatim}
perm_result = permutation_importance(
    rf, X, y,
    n_repeats=10,
    random_state=42,
    n_jobs=-1
)

perm_importance = pd.DataFrame({
    'feature': feature_cols,
    'importance': perm_result.importances_mean,
    'std': perm_result.importances_std
})
\end{verbatim}

Ten permutation repeats provide stable estimates with quantified uncertainty.

\section{Visualisation Implementation}

Feature importance plots were generated using matplotlib:

\begin{verbatim}
fig, ax = plt.subplots(figsize=(12, 8))
top20 = perm_importance.head(20)

bars = ax.barh(range(len(top20)), top20['importance'].values)
ax.set_yticks(range(len(top20)))
ax.set_yticklabels(top20['feature'].values)
ax.invert_yaxis()  # Highest importance at top
ax.set_xlabel('Permutation Importance')
ax.set_title(f'Top 20 Features - {task.title()} Task')

plt.savefig(f'figures/{task}_feature_importance.png', dpi=150)
\end{verbatim}

Horizontal bar charts facilitate reading long feature names. DPI of 150 balances file size and print quality.

\section{Output Management}

\subsection{Results Storage}

All results are saved in structured directories:

\begin{verbatim}
results/
├── summary.csv              # Overall accuracy comparison
├── reading_gini_importance.csv
├── reading_perm_importance.csv
├── interview_gini_importance.csv
└── interview_perm_importance.csv

figures/
├── reading_feature_importance.png
└── interview_feature_importance.png
\end{verbatim}

\subsection{Version Control}

Git tracks all code and configuration changes. Large binary files (audio corpus, compressed archives) are excluded via \texttt{.gitignore}:

\begin{verbatim}
# .gitignore
data/Androids-Corpus/
data/*.zip
.venv/
__pycache__/
\end{verbatim}

\section{Execution Workflow}

The complete analysis can be reproduced with two commands:

\begin{verbatim}
# 1. Extract features (run once)
python scripts/extract_features.py

# 2. Run classification and importance analysis
python scripts/run_analysis.py
\end{verbatim}

Feature extraction takes approximately 2-3 minutes for 228 audio files. Classification and importance analysis completes in under 1 minute.

\section{Summary}

The implementation follows software engineering best practices:

\begin{itemize}
    \item \textbf{Modularity:} Separate scripts for extraction and analysis
    \item \textbf{Reproducibility:} Fixed random seeds, virtual environment, version control
    \item \textbf{Efficiency:} Parallel processing where possible
    \item \textbf{Documentation:} Clear variable names, inline comments
\end{itemize}

All code is available in the project repository, enabling full replication of results.


\chapter{Results}

This chapter presents the experimental results, including classification performance, confusion matrix analysis, statistical significance testing, feature importance analysis, error analysis, and comparison between speech tasks.

\section{Classification Performance}

\subsection{Overall Results}

Table \ref{tab:classification-results} summarises the classification performance for both speech tasks using 5-fold stratified cross-validation. The reported uncertainty ($\pm$) represents the standard deviation across the five cross-validation folds, reflecting variability in performance estimates.

\begin{table}[h]
\centering
\caption{Classification results by task and algorithm}
\label{tab:classification-results}
\begin{tabular}{lcccc}
\hline
\textbf{Task} & \textbf{Algorithm} & \textbf{Accuracy} & \textbf{F1 Score} \\
\hline
Reading & SVM & 71.5\% $\pm$ 6.8\% & 0.73 \\
Reading & Random Forest & 72.3\% $\pm$ 7.2\% & 0.74 \\
\hline
Interview & SVM & 82.0\% $\pm$ 5.2\% & 0.85 \\
Interview & Random Forest & \textbf{87.1\%} $\pm$ 4.8\% & \textbf{0.88} \\
\hline
\end{tabular}
\end{table}

For context, with approximately balanced classes (48\% HC, 52\% PT), chance-level accuracy is approximately 50\%. All classifiers substantially exceed this baseline, confirming that the acoustic features carry genuine discriminative information about depression status.

\subsection{Comparison with Prior Work}

These results compare favourably with Tao et al.'s published results on the same ANDROIDS corpus \cite{Tao2024}. Using deep learning approaches, they achieved 83.4\% accuracy on reading and 81.6\% on spontaneous speech. The present work achieves 87.1\% on interview speech using traditional machine learning---a 5.5 percentage point improvement over their spontaneous speech result, while maintaining full interpretability. This validates the argument that interpretable methods can be competitive with deep learning for this task.

\subsection{Key Finding: Interview Outperforms Reading}

The most striking result is the substantial performance gap between tasks. Interview speech yields approximately 15 percentage points higher accuracy than reading speech across both classifiers:

\begin{itemize}
    \item \textbf{Reading task:} 71.5--72.3\% accuracy
    \item \textbf{Interview task:} 82.0--87.1\% accuracy
\end{itemize}

This suggests that spontaneous speech contains richer markers of depression than controlled reading, likely because it captures a broader range of cognitive and emotional processes.

\subsection{Statistical Significance}

To confirm that this performance difference is genuine and not due to random variation, a two-proportion z-test was conducted comparing the Random Forest accuracy rates across tasks.

\begin{table}[h]
\centering
\caption{Statistical significance test: Reading vs Interview}
\label{tab:significance}
\begin{tabular}{lc}
\hline
\textbf{Metric} & \textbf{Value} \\
\hline
Reading accuracy & 72.3\% \\
Interview accuracy & 87.1\% \\
Difference & 14.8 percentage points \\
Z-statistic & 2.774 \\
P-value & 0.0055 \\
\hline
\end{tabular}
\end{table}

The p-value of 0.0055 is well below the conventional significance threshold of 0.05, indicating that \textbf{interview speech is statistically significantly better than reading speech for depression detection}.

\subsection{Classifier Comparison}

Random Forest marginally outperformed SVM on both tasks:
\begin{itemize}
    \item Reading: RF 72.3\% vs SVM 71.5\% (+0.8 percentage points)
    \item Interview: RF 87.1\% vs SVM 82.0\% (+5.1 percentage points)
\end{itemize}

The consistency of the interview $>$ reading pattern across both classifiers suggests this finding is robust to algorithm choice rather than an artefact of a particular method.

\section{Confusion Matrix Analysis}

Confusion matrices provide detailed insight into classification errors. Tables \ref{tab:cm-reading} and \ref{tab:cm-interview} show the confusion matrices for Random Forest on each task.

\begin{table}[h]
\centering
\caption{Confusion matrix: Reading task (Random Forest)}
\label{tab:cm-reading}
\begin{tabular}{lcc}
\hline
 & \textbf{Predicted HC} & \textbf{Predicted PT} \\
\hline
\textbf{Actual HC} & 36 & 18 \\
\textbf{Actual PT} & 13 & 45 \\
\hline
\end{tabular}
\end{table}

\begin{table}[h]
\centering
\caption{Confusion matrix: Interview task (Random Forest)}
\label{tab:cm-interview}
\begin{tabular}{lcc}
\hline
 & \textbf{Predicted HC} & \textbf{Predicted PT} \\
\hline
\textbf{Actual HC} & 44 & 8 \\
\textbf{Actual PT} & 7 & 57 \\
\hline
\end{tabular}
\end{table}

Key observations:

\paragraph{Reading Task:}
\begin{itemize}
    \item False Positives (HC $\rightarrow$ PT): 18 (33\% of healthy controls)
    \item False Negatives (PT $\rightarrow$ HC): 13 (22\% of patients)
    \item The model shows a slight bias toward predicting depression
\end{itemize}

\paragraph{Interview Task:}
\begin{itemize}
    \item False Positives: 8 (15\% of healthy controls)
    \item False Negatives: 7 (11\% of patients)
    \item Substantially improved error rates with more balanced errors
\end{itemize}

The interview task confusion matrix demonstrates not only higher accuracy but also more balanced error types---both false positives and false negatives are low. This is clinically important: false positives cause unnecessary concern while false negatives mean missed cases.

\subsection{Detailed Classification Metrics}

Table \ref{tab:detailed-metrics} presents precision, recall, and F1 scores for each class.

\begin{table}[h]
\centering
\caption{Detailed classification metrics (Random Forest)}
\label{tab:detailed-metrics}
\begin{tabular}{llccc}
\hline
\textbf{Task} & \textbf{Class} & \textbf{Precision} & \textbf{Recall} & \textbf{F1} \\
\hline
Reading & Healthy & 0.73 & 0.67 & 0.70 \\
Reading & Depressed & 0.71 & 0.78 & 0.74 \\
\hline
Interview & Healthy & 0.86 & 0.85 & 0.85 \\
Interview & Depressed & 0.88 & 0.89 & 0.88 \\
\hline
\end{tabular}
\end{table}

\section{Feature Importance Analysis}

The primary research question concerns which acoustic features are most predictive of depression. This section presents feature importance results ranked by Gini importance from the Random Forest classifier. Permutation importance was also computed as a robustness check; the two measures showed broad agreement in identifying the most predictive features, increasing confidence in these rankings.

\subsection{Reading Task Features}

Table \ref{tab:reading-features} shows the top 10 features for the reading task.

\begin{table}[h]
\centering
\caption{Top 10 predictive features for reading task}
\label{tab:reading-features}
\begin{tabular}{clc}
\hline
\textbf{Rank} & \textbf{Feature} & \textbf{Importance} \\
\hline
1 & slopeUV0-500\_amean & 0.050 \\
2 & mfcc1\_amean & 0.042 \\
3 & mfcc1\_stddevNorm & 0.041 \\
4 & loudnessPeaksPerSec & 0.036 \\
5 & StddevVoicedSegmentLengthSec & 0.032 \\
6 & mfcc1V\_amean & 0.029 \\
7 & slopeV500-1500\_stddevNorm & 0.027 \\
8 & VoicedSegmentsPerSec & 0.026 \\
9 & slopeV500-1500\_amean & 0.025 \\
10 & F2amplitudeLogRelF0\_amean & 0.025 \\
\hline
\end{tabular}
\end{table}

Importance scores range from 0.050 (top feature) to 0.025 (10th feature), indicating a relatively gradual decline. The top three features account for approximately 13\% of total importance, suggesting predictive power is distributed across multiple features rather than concentrated in one or two.

\subsubsection{Interpretation: Reading Task}

\paragraph{Spectral Slope (slopeUV0-500)} The top feature measures spectral tilt in unvoiced regions (0--500 Hz). Flatter slopes indicate breathier, less energetic voice quality---consistent with reduced vocal effort in depression.

\paragraph{MFCC1} The first mel-frequency cepstral coefficient captures overall spectral envelope shape. Both mean and variability appear in the top 10, suggesting depressed speech shows altered spectral characteristics.

\paragraph{Temporal Features} Loudness peaks per second and voiced segment variability reflect speech rhythm and prosodic patterns.

\subsection{Interview Task Features}

Table \ref{tab:interview-features} shows the top 10 features for the interview task.

\begin{table}[h]
\centering
\caption{Top 10 predictive features for interview task}
\label{tab:interview-features}
\begin{tabular}{clc}
\hline
\textbf{Rank} & \textbf{Feature} & \textbf{Importance} \\
\hline
1 & spectralFluxV\_stddevNorm & 0.069 \\
2 & hammarbergIndexV\_stddevNorm & 0.043 \\
3 & StddevUnvoicedSegmentLength & 0.038 \\
4 & MeanUnvoicedSegmentLength & 0.035 \\
5 & alphaRatioV\_stddevNorm & 0.035 \\
6 & mfcc1V\_amean & 0.033 \\
7 & mfcc1V\_stddevNorm & 0.032 \\
8 & loudness\_stddevRisingSlope & 0.030 \\
9 & loudness\_stddevFallingSlope & 0.026 \\
10 & logRelF0-H1-A3\_amean & 0.020 \\
\hline
\end{tabular}
\end{table}

Importance scores show a steeper decline than the reading task: the top feature (0.069) is nearly 3.5 times more important than the 10th feature (0.020). This suggests that a smaller set of features dominates predictive power for spontaneous speech.

\subsubsection{Interpretation: Interview Task}

\paragraph{Spectral Flux Variability} The most important feature measures frame-to-frame spectral change variability. Reduced variability indicates monotonous speech---a hallmark of depression-related flat affect.

\paragraph{Voice Quality Variability} The Hammarberg index and alpha ratio capture spectral balance related to breathiness and vocal strain. Notably, their \textit{variability} (standard deviation) rather than mean values drives predictions.

\paragraph{Pause Characteristics} Both mean and variability of unvoiced segment length rank highly. Unvoiced segments correspond to pauses and hesitations---clinically meaningful given the psychomotor retardation associated with depression.

\subsection{Key Insight: Variability Measures Dominate}

A striking pattern emerges: for interview speech, 7 of the top 10 features are variability measures (standard deviations), while for reading speech, only 2 of the top 10 are variability measures. This aligns with clinical characterisations of depressed speech as ``flat'' or ``monotonous''---not necessarily different in average properties, but reduced in \textit{dynamic modulation}.

\section{Task Comparison}

\subsection{Feature Overlap}

Comparing the top 10 features between tasks reveals minimal overlap: exactly \textbf{one feature} (mfcc1V\_amean) appears in both lists. This suggests that different acoustic markers are salient depending on speech context---a finding with implications for clinical system design.

\subsection{Dominant Feature Categories by Task}

\begin{table}[h]
\centering
\caption{Dominant feature categories by task}
\label{tab:feature-categories-comparison}
\begin{tabular}{lll}
\hline
\textbf{Category} & \textbf{Reading} & \textbf{Interview} \\
\hline
Spectral & Slope, MFCC1 & Flux, MFCC1V \\
Voice Quality & -- & Hammarberg, Alpha ratio \\
Temporal & Voiced segments & Unvoiced segments (pauses) \\
Prosodic & Loudness peaks & Loudness dynamics \\
\hline
\end{tabular}
\end{table}

\subsection{Clinical Interpretation}

The task-specific feature profiles suggest different cognitive mechanisms are captured:

\paragraph{Reading Task} Reading scripted text primarily reveals \textit{voice production} characteristics (spectral slope, MFCC) and basic rhythm. These may reflect the physiological and motor aspects of depression.

\paragraph{Interview Task} Spontaneous speech reveals \textit{cognitive and affective} processes---variable pausing (reflecting cognitive load and word-finding difficulty), reduced vocal dynamics (flat affect), and voice quality changes (emotional expression). The interview task places greater demands on executive function, emotional regulation, and language production---all impacted by depression.

This explains the superior classification performance on interview speech: it captures a broader range of depression-related processes.

\section{Error Analysis}

\subsection{Misclassification by Gender}

Analysis of misclassified samples revealed a pattern related to gender. To interpret this correctly, error \textit{rates} (not raw counts) must be considered given the gender imbalance in the dataset.

\begin{table}[h]
\centering
\caption{Misclassification rates by gender}
\label{tab:error-gender}
\begin{tabular}{llccc}
\hline
\textbf{Task} & \textbf{Gender} & \textbf{Errors} & \textbf{Total} & \textbf{Error Rate} \\
\hline
Reading & Female & 23 & 80 & 28.7\% \\
Reading & Male & 8 & 32 & 25.0\% \\
Interview & Female & 12 & 84 & 14.3\% \\
Interview & Male & 3 & 32 & 9.4\% \\
\hline
\end{tabular}
\end{table}

Female speakers show slightly higher error rates in both tasks (28.7\% vs 25.0\% for reading; 14.3\% vs 9.4\% for interview). However, these differences are modest---the raw count disparity (23 vs 8) is largely explained by the dataset being approximately 70\% female. This finding warrants further investigation but does not undermine the main results, since both genders show the same pattern of interview speech outperforming reading speech for depression detection.

\subsection{Validation Checks}

Two checks confirm the results are not artefactual:

\paragraph{No overfitting detected} Learning curves (Figure \ref{fig:learning-curves}) show training and validation scores converging, indicating the models generalise appropriately. Training accuracy reached approximately 92\% while validation accuracy stabilised around 87\% for the interview task.

\paragraph{Cross-method consistency} Both SVM and Random Forest produce the same pattern (interview $>$ reading), suggesting the finding is robust to algorithm choice.

\section{Visualisations}

Figure \ref{fig:feature-importance} presents the feature importance distributions for both tasks. The visual contrast is striking: reading task features show a relatively even distribution, while interview task features exhibit a steep drop-off after the top few features. Spectral flux variability visually dominates the interview chart, reinforcing its role as the primary predictive feature for spontaneous speech.

\begin{figure}[h]
    \centering
    \includegraphics[width=\textwidth]{figures/advanced/feature_importance_comparison.png}
    \caption{Top 10 features by Gini importance: Reading task (left) vs Interview task (right). Note the steeper importance gradient for interview speech.}
    \label{fig:feature-importance}
\end{figure}

Figure \ref{fig:confusion-matrices} shows confusion matrix heatmaps for both tasks. The reading task matrix shows moderate off-diagonal values (errors), while the interview task matrix is visually cleaner---darker diagonal cells and lighter off-diagonal cells---immediately communicating the improved classification performance.

\begin{figure}[h]
    \centering
    \includegraphics[width=0.48\textwidth]{figures/advanced/reading_confusion_matrices.png}
    \includegraphics[width=0.48\textwidth]{figures/advanced/interview_confusion_matrices.png}
    \caption{Confusion matrices: Reading task (left) and Interview task (right). The interview task shows improved classification with fewer off-diagonal errors.}
    \label{fig:confusion-matrices}
\end{figure}

Figure \ref{fig:learning-curves} tracks training and cross-validation accuracy as training set size increases. Both tasks show the characteristic pattern of valid learning: training accuracy starts high and decreases slightly as training size grows, while validation accuracy increases and converges toward the training curve. The modest gap between curves at maximum training size ($\sim$5 percentage points) indicates appropriate generalisation without severe overfitting.

\begin{figure}[h]
    \centering
    \includegraphics[width=0.48\textwidth]{figures/advanced/reading_learning_curve.png}
    \includegraphics[width=0.48\textwidth]{figures/advanced/interview_learning_curve.png}
    \caption{Learning curves showing convergence of training and validation accuracy, indicating no severe overfitting.}
    \label{fig:learning-curves}
\end{figure}

\section{Summary of Findings}

\begin{enumerate}
    \item \textbf{Interview speech is significantly more informative:} 87.1\% vs 72.3\% accuracy (p = 0.0055), exceeding prior deep learning results on the same corpus
    
    \item \textbf{Different features matter for each task:} Only 1 of 10 top features overlaps between tasks
    \begin{itemize}
        \item Reading: Spectral slope, MFCC, rhythm
        \item Interview: Spectral dynamics, pausing, voice quality variability
    \end{itemize}
    
    \item \textbf{Variability measures are key for interview speech:} 7 of 10 top features are standard deviations, consistent with ``flat'' depressed speech
    
    \item \textbf{Pausing behaviour is highly informative:} Unvoiced segment features rank 3rd and 4th for interview speech
    
    \item \textbf{Confusion matrices show balanced errors for interview:} Both false positives (8) and false negatives (7) are low
    
    \item \textbf{Gender shows modest effect on error rates:} Female speakers have slightly higher error rates, though the difference is small
\end{enumerate}

\subsection{Limitations}

Feature importance rankings identify \textit{which} features are predictive but do not establish whether differences between features are statistically significant. Future work could apply permutation tests or bootstrap confidence intervals to determine whether, for example, the top-ranked feature is significantly more important than the second-ranked feature. For present purposes, the consistency between Gini and permutation importance measures provides reasonable confidence in the top-ranked features.


\chapter{Discussion}

This chapter interprets the experimental findings, examines their validity, situates them within the broader literature, explores their implications, acknowledges limitations, and identifies directions for future research.

\section{Interpretation of Key Findings}

\subsection{Why Interview Speech Outperforms Reading}

The 15-percentage-point accuracy advantage of interview over reading speech is this study's most significant finding. This is not merely a methodological detail—it suggests that the choice of speech elicitation task fundamentally shapes what depression markers can be detected. Understanding \textit{why} this gap exists illuminates both the nature of depressed speech and the design of future screening systems.

The most compelling explanation involves cognitive load. Spontaneous speech requires concurrent planning, lexical retrieval, syntactic structuring, and articulation—processes that draw heavily on executive function and working memory. Depression impairs both of these cognitive domains \cite{rock2014cognitive}. When a depressed individual must simultaneously decide \textit{what} to say and \textit{how} to say it, these impairments manifest in the acoustic signal: longer pauses during word-finding, reduced prosodic variation as cognitive resources are diverted from expressive modulation, and increased hesitations. Reading eliminates most of this cognitive demand by providing the linguistic content directly. The speaker need only decode and articulate, leaving fewer channels through which depression-related deficits can emerge.

Emotional engagement provides a complementary explanation. The ANDROIDS interview protocol asks participants about daily routines, social relationships, and future plans—topics that inherently engage affective processing. Depression fundamentally alters emotional experience and expression, but these alterations may only surface when emotionally relevant content is being discussed. Neutral reading passages, by contrast, provide no affective context to which a depressed individual might respond differently than a healthy control.

The prominence of pause-related features in interview speech supports a third mechanism: spontaneous speech uniquely captures the dynamics of thought-to-speech translation. The feature \texttt{StddevUnvoicedSegmentLength}—variability in pause duration—ranks among the top predictors for interview speech but not for reading. This reflects the variable latencies involved in response formulation, lexical access, and self-monitoring. In depressed speakers, psychomotor retardation and slowed cognitive processing elongate these pauses unpredictably. Fluent reading simply does not permit such pauses to emerge.

Finally, spontaneous speech allows natural prosodic variability that reading constrains. A reader follows the cadence suggested by punctuation and syntax; a conversational speaker modulates pitch, loudness, and timing according to communicative intent and emotional state. This variability is itself diagnostic. The finding that variability measures dominate interview speech classification suggests that depression's acoustic signature is not a shift in average voice characteristics but a \textit{compression of dynamic range}—the ``flat'' or ``monotonous'' quality clinicians have long observed.

These mechanisms likely operate jointly. Interview speech simultaneously increases cognitive load, engages emotional processing, permits natural pausing, and allows prosodic freedom—each channel providing opportunities for depression-related differences to emerge. Reading suppresses all four.

\subsection{The Dominance of Variability Measures}

Perhaps the most striking pattern in the results is the dominance of variability measures for interview speech classification. The top three features are all standard deviations:

\begin{enumerate}
    \item \texttt{spectralFluxV\_stddevNorm} — variability in spectral change rate
    \item \texttt{hammarbergIndexV\_stddevNorm} — variability in voice quality (spectral tilt)
    \item \texttt{StddevUnvoicedSegmentLength} — variability in pause durations
\end{enumerate}

This finding has important implications. It suggests that depression does not simply shift the voice to a different register or quality—depressed speakers do not merely have a lower pitch or quieter voice on average. Rather, depression appears to reduce the \textit{modulation} of these properties over time. A healthy speaker varies their spectral characteristics dynamically during conversation; a depressed speaker exhibits reduced range.

This aligns with clinical descriptions of depressed speech as ``monotonous'' or lacking in affective prosody, but provides a more precise characterisation. The clinical intuition is correct, but the mechanism is specific: it is the temporal dynamics, not the static properties, that carry diagnostic information. This has practical implications for screening system design—algorithms should prioritise measures of change over time rather than summary statistics like means.

The contrast with reading speech is instructive. For reading, mean values (e.g., \texttt{slopeUV0-500\_amean}, \texttt{mfcc1\_amean}) are more predictive than variability measures. This makes sense: reading constrains natural prosodic variation, so the variability channel carries less information. What remains is the speaker's baseline vocal characteristics, and here depression may manifest as altered spectral slopes or vocal tract configurations.

\section{Validity and Robustness of Results}

An interview speech accuracy of 87.1\% invites healthy scepticism. Before interpreting this result, we must consider whether it reflects genuine predictive power or methodological artefact.

Several lines of evidence support validity. First, the learning curves (Figure 5.3) show no evidence of severe overfitting. Training and validation curves converge appropriately as sample size increases, and the gap between them remains modest. If the model were memorising training data rather than learning generalisable patterns, we would expect the training curve to plateau at near-perfect accuracy while validation accuracy remained low.

Second, both classifiers—SVM and Random Forest—show the same pattern. Interview speech substantially outperforms reading speech regardless of classifier choice (SVM: 82.0\% vs 71.5\%; Random Forest: 87.1\% vs 72.3\%). If the result were an artefact of a particular classifier's idiosyncrasies, we would not expect this consistency.

Third, the features identified as most predictive are clinically interpretable. Spectral flux variability, voice quality modulation, and pause characteristics all have documented relationships with depression in the clinical literature. If the classifier were exploiting spurious correlations—recording artefacts, environmental noise, or demographic confounds—we would expect arbitrary features to dominate. Instead, the top features map onto established clinical observations about depressed speech.

Fourth, as discussed below, the results compare favourably with prior work on the same corpus using more complex methods. Tao's PhD thesis \cite{tao2020phd} applied deep learning to ANDROIDS data and achieved comparable or lower accuracy on similar tasks. Our simpler, interpretable approach matches or exceeds this benchmark, suggesting the signal is robust rather than requiring complex models to extract.

Finally, the statistical significance of the task difference (z = 2.78, p = 0.0055) provides confidence that the interview-reading gap is not a chance fluctuation. Even under conservative assumptions about the independence of cross-validation folds, this difference is unlikely to have arisen by chance.

\section{Relation to Prior Literature}

\subsection{Convergence with Established Findings}

The results align with and extend established findings in the depression-speech literature. The importance of prosodic variability measures converges with decades of clinical observation and empirical research documenting reduced pitch range and loudness variation in depressed speech \cite{cummins2015review, low2011automated}. The prominence of pause-related features reflects psychomotor retardation, a core symptom of melancholic depression that has been reliably linked to speech timing characteristics \cite{sobin1997psychomotor}. The predictive value of voice quality indicators like the Hammarberg index corroborates studies documenting breathiness and altered phonation in depression \cite{scherer1986vocal}.

However, the most interesting comparisons involve specific features rather than broad patterns. The emergence of \texttt{spectralFluxV\_stddevNorm} as the top interview feature deserves particular attention. Spectral flux—the rate of change in the spectral envelope—has been explored in music information retrieval and emotion recognition, but its prominence in depression detection is less established. A review of recent depression-speech literature finds few studies highlighting spectral flux specifically. \cite{alghowinem2013comparative} examined a range of spectral features but focused primarily on MFCCs and formants. \cite{cummins2015review} catalogued commonly studied features without emphasising spectral flux. 

Why might this study surface spectral flux when others have not? One possibility is the specific focus on spontaneous speech with adequate duration. Spectral flux captures rapid timbral changes that occur during natural conversation—shifts in phonation, articulatory transitions, emotional colouring. In short prompted utterances or reading tasks, there may simply be insufficient variation for spectral flux to be informative. The ANDROIDS interview recordings, which average several minutes of genuine conversation, provide ample opportunity for these dynamics to emerge.

\subsection{Divergence and Novel Contributions}

Several findings extend or challenge prior work.

\textbf{The magnitude of the task effect.} While researchers have noted that different speech tasks yield different results, few studies have quantified this directly using the same participants. The 15-percentage-point accuracy gap is substantial—larger than the improvements typically achieved through algorithmic refinements. This suggests that task selection may matter more than classifier optimisation for achieving high detection rates.

\textbf{The minimal feature overlap between tasks.} Of the top 10 features for each task, only one (\texttt{F1amplitudeLogRelF0\_stddevNorm}) appears in both lists. This near-complete divergence suggests that reading and interview tasks do not simply provide stronger or weaker versions of the same signal—they provide access to \textit{qualitatively different} information about depression. Reading task classification relies primarily on static spectral characteristics (MFCCs, spectral slopes); interview task classification relies on dynamic modulation (variability in flux, voice quality, pausing). Prior literature has generally assumed that depression affects speech in a unified way that different tasks capture with varying fidelity. These results suggest a more nuanced picture: depression may affect spontaneous and constrained speech through partially distinct mechanisms.

\textbf{Competitive performance with interpretable methods.} The 87.1\% accuracy achieved by Random Forest on interview speech matches or exceeds results reported by studies using deep learning on comparable data. Tao's PhD thesis \cite{tao2020phd}, which applied convolutional and recurrent neural networks to the ANDROIDS corpus, reported similar classification performance. Yet those approaches sacrifice interpretability—the models provide predictions but not explanations. The present study demonstrates that traditional machine learning with domain-informed features can achieve equivalent accuracy while preserving the ability to identify which specific acoustic properties drive detection. For clinical applications where trust and explainability matter, this is a meaningful advantage.

\subsection{Contextualising Performance: AVEC and Beyond}

Comparing results across studies requires care. The Audio/Visual Emotion Challenge (AVEC) series established influential benchmarks for speech-based depression detection, with reported accuracies typically ranging from 70\% to 85\% depending on the specific task and evaluation metric \cite{valstar2013avec, ringeval2017avec}.

However, direct comparison with AVEC results is complicated by several factors. AVEC used the DAIC-WOZ corpus, which differs from ANDROIDS in population (American vs. Italian), labelling scheme (PHQ-8 self-report vs. clinical diagnosis), and recording context (Wizard-of-Oz interview vs. human interviewer). Many AVEC submissions employed multimodal approaches combining audio, video, and text, while this study used acoustic features alone. AVEC also used regression to predict depression severity scores rather than binary classification.

With these caveats, the present results are encouraging. An 87\% binary classification accuracy using only acoustic features on clinically-diagnosed participants compares favourably with multimodal systems predicting self-reported symptoms. This suggests that clean clinical labels and appropriate task selection may be as valuable as complex multimodal architectures.

\section{Implications}

\subsection{For Clinical Practice}

The results have several practical implications for speech-based depression screening.

\textbf{Task selection matters more than often recognised.} If a clinician or screening system must choose one speech elicitation task, spontaneous speech should be strongly preferred. The 15-point accuracy advantage is not marginal—it represents the difference between a tool that is clinically useful and one that is not. Asking a patient to describe their typical day, discuss their mood, or recount a recent experience is likely to yield more diagnostic information than having them read a standardised passage.

\textbf{Variability measures deserve priority.} Current clinical intuitions about ``flat affect'' or ``monotonous speech'' are supported by these findings, but the specific recommendation is to measure \textit{temporal variability} in acoustic features rather than their means. A screening algorithm should track how much a speaker's voice quality, spectral characteristics, and pausing patterns fluctuate over the course of a conversation.

\textbf{Interpretable methods remain viable.} The success of Random Forest classification suggests that black-box deep learning is not required for competitive performance. For clinical settings where explainability supports trust and adoption, traditional machine learning with meaningful features may be preferable.

\subsection{For Research}

\textbf{Single-task studies may miss important markers.} If reading and interview tasks access different depression-related information, studies using only one task may reach incomplete conclusions. A finding that a particular feature is not predictive of depression may simply reflect the wrong task context.

\textbf{Feature importance is task-contingent.} Researchers should be cautious about claims like ``MFCCs are the most important features for depression detection.'' This may be true for reading tasks but not for spontaneous speech. The literature would benefit from more systematic task comparisons.

\textbf{The nature of depressed speech may be more nuanced than assumed.} The common model—that depression produces a characteristic set of acoustic biomarkers—may need refinement. Depression may affect speech differently depending on communicative context, with constrained tasks revealing static voice changes and spontaneous tasks revealing dynamic modulation deficits.

\subsection{For Screening System Design}

Practical depression screening tools should incorporate these insights:

\begin{itemize}
    \item Elicit spontaneous speech through open-ended questions rather than scripted prompts
    \item Compute variability measures (standard deviations, ranges) alongside means
    \item Consider task-specific models rather than one-size-fits-all classifiers
    \item Prioritise clinically interpretable features to support validation and trust
\end{itemize}

\section{Limitations}

The findings must be interpreted within several constraints, ranked here by likely impact on conclusions.

\subsection{Most Serious Limitations}

\textbf{Language specificity.} The ANDROIDS corpus contains exclusively Italian speech. Prosodic patterns, pause behaviour, and spectral characteristics are known to vary across languages. The specific features identified as predictive may not transfer to English, Mandarin, or other languages. Cross-linguistic validation is essential before these findings can inform tools for non-Italian populations. This is the most significant threat to generalisability.

\textbf{Sample size and diversity.} With 118 unique speakers and 228 recordings, the dataset is modest by machine learning standards. The results may not capture the full variability of depressed speech, and minority patterns or subtypes may be underrepresented. Additionally, the corpus does not stratify by depression severity (mild vs. moderate vs. severe) or subtype (melancholic vs. atypical vs. anxious). Different presentations may have distinct acoustic signatures that a pooled analysis cannot detect.

\subsection{Moderate Limitations}

\textbf{Gender imbalance and potential bias.} The corpus is 72\% female. Error analysis revealed that women were more frequently misclassified than men in both tasks (reading: 23 female vs. 8 male errors; interview: 12 female vs. 3 male errors). This disparity persists even accounting for the higher proportion of female speakers. Several interpretations are possible: depression may manifest differently in male and female speech, the training data may contain gender-specific confounds, or the smaller male sample may be inadequately modelled. This finding has important equity implications for deployed screening systems—a tool trained on imbalanced data may perform worse for underrepresented groups.

\textbf{Cross-sectional design.} Recordings capture a single time point per participant. This study cannot assess whether speech markers track symptom changes over time, respond to treatment, or predict future episodes. Longitudinal validation is needed for monitoring applications.

\textbf{No speaker normalisation.} Features were extracted without adjusting for individual baseline voice characteristics. Speaker adaptation techniques might improve robustness, particularly for applications where the same individual is monitored over time.

\subsection{Minor Limitations}

\textbf{Feature set constraints.} The eGeMAPS set, while comprehensive and standardised, may not capture all relevant acoustic information. Alternative representations (raw spectrograms, learned embeddings, linguistic features) might provide complementary information.

\textbf{Classifier scope.} Only SVM and Random Forest were evaluated. Gradient boosting, attention-based models, or ensemble methods might achieve modestly better performance, though the consistency between SVM and Random Forest suggests the main findings are robust to classifier choice.

\textbf{Independence assumption in statistical test.} The z-test for comparing task accuracies assumes independent samples. Because the same participants appear in both tasks, this assumption is technically violated. However, the highly significant p-value (0.0055) provides a margin of safety—even under more conservative tests, the task difference would likely remain significant.

\section{Future Work}

\subsection{Immediate Extensions}

\textbf{SHAP analysis.} The planned SHapley Additive exPlanations analysis would provide instance-level feature attributions, showing which features drive predictions for individual recordings. This could identify whether subgroups of depressed speakers have distinct acoustic profiles—potentially revealing depression subtypes.

\textbf{Cross-corpus validation.} Testing on the DAIC-WOZ corpus or other publicly available datasets would assess generalisability. Transfer learning approaches could adapt models trained on one corpus to another, potentially overcoming language and recording condition differences.

\textbf{Gender-stratified analysis.} Given the observed error disparity, training separate models for male and female speakers—or including gender-aware normalisation—could improve equity and overall performance.

\subsection{Longer-Term Directions}

\textbf{Longitudinal monitoring.} Tracking speech markers over the course of treatment could provide objective measures of response, complementing self-report questionnaires. Early evidence suggests speech features may change before subjective symptom reports, offering potential for early detection of relapse or recovery.

\textbf{Multimodal integration.} Combining acoustic features with linguistic content (word choice, sentiment, coherence), visual cues (facial expression, gesture), or physiological signals could improve detection rates and provide richer clinical profiles.

\textbf{Mechanism investigation.} This study identifies predictive features but not causal mechanisms. Controlled experiments manipulating cognitive load, emotional content, or task structure could clarify \textit{why} certain features are informative, supporting theory development alongside practical applications.

\textbf{Real-world deployment studies.} Laboratory accuracy does not guarantee clinical utility. Pilot studies in clinical settings—testing acceptance, workflow integration, and real-world performance—are necessary before speech-based screening can be responsibly deployed.

\section{Summary}

This study demonstrates that spontaneous interview speech substantially outperforms reading tasks for depression detection, achieving 87\% accuracy with interpretable acoustic features. The most predictive features are variability measures—particularly spectral flux, voice quality, and pause duration variability—suggesting that depression compresses the dynamic range of speech rather than shifting its average properties.

The results are robust across classifiers, supported by appropriate learning curves, and consistent with clinical observations about depressed speech. Comparison with prior work suggests that interpretable traditional methods can match deep learning performance while providing clinical insight.

Key implications include the importance of task selection for screening system design, the value of variability measures over means, and the finding that reading and interview tasks access qualitatively different information about depression. Limitations regarding language specificity, sample size, and gender bias must be addressed through cross-corpus validation and more diverse data collection.

Future work should pursue SHAP analysis for instance-level explanation, cross-corpus validation for generalisability assessment, and longitudinal studies for clinical monitoring applications. The findings establish a foundation for speech-based depression assessment while highlighting the nuanced relationship between speech elicitation context and diagnostic information.


\chapter{Conclusion}

\section{Summary of Contributions}

This dissertation investigated which acoustic features of speech are most predictive of depression, with particular focus on comparing read versus spontaneous speech modalities. Using the ANDROIDS corpus and interpretable machine learning methods, the study addressed the research question directly and yielded several key contributions:

\subsection{Research Question Answered}

The research question asked: \textit{``Which acoustic features of speech are most predictive of depression, and how do they differ between read and spontaneous speech?''}

The answer is nuanced and task-dependent:

\paragraph{For Reading Speech:}
\begin{itemize}
    \item Spectral slope features (particularly in the 0--500 Hz range)
    \item First Mel-Frequency Cepstral Coefficient (MFCC1)
    \item Loudness peaks per second
    \item Voiced segment timing variability
\end{itemize}

\paragraph{For Interview Speech:}
\begin{itemize}
    \item Spectral flux variability
    \item Voice quality measures (Hammarberg index, alpha ratio)
    \item Pause characteristics (unvoiced segment duration and variability)
    \item Loudness dynamics (rising and falling slopes)
\end{itemize}

The critical finding is that \textbf{variability measures dominate for spontaneous speech}, while \textbf{mean values are more important for reading tasks}.

\subsection{Main Findings}

\begin{enumerate}
    \item \textbf{Interview speech substantially outperforms reading speech} for depression detection (87\% vs 72\% accuracy), suggesting that spontaneous speech captures richer markers of depression.
    
    \item \textbf{Different acoustic features are predictive in different speech contexts}, indicating that task selection is a critical design decision for speech-based mental health assessment.
    
    \item \textbf{Interpretable methods achieve competitive performance} with published deep learning approaches while providing clinical insight into which speech characteristics are affected by depression.
    
    \item \textbf{Dynamic speech modulation is key}: the most predictive features for interview speech are variability measures, consistent with clinical descriptions of ``flat'' or ``monotonous'' depressed speech.
    
    \item \textbf{Pause behaviour is highly informative}, reflecting the psychomotor retardation and cognitive slowing associated with depression.
\end{enumerate}

\subsection{Practical Implications}

For developers of speech-based mental health tools:
\begin{itemize}
    \item \textbf{Elicit spontaneous speech} rather than scripted reading for maximum diagnostic value
    \item \textbf{Prioritise variability features} when building detection models
    \item \textbf{Consider pause analysis} as a computationally efficient and interpretable marker
\end{itemize}

For clinicians:
\begin{itemize}
    \item Speech characteristics may provide objective biomarkers complementing self-report
    \item Reduced vocal dynamics and altered pausing are observable indicators worth clinical attention
\end{itemize}

\section{Limitations Acknowledged}

This work has limitations that constrain generalisation:

\begin{itemize}
    \item The findings are derived from a single corpus (ANDROIDS) with Italian speakers; cross-linguistic and cross-cultural validation is needed
    \item Sample size (228 recordings) is modest for machine learning
    \item The study focused on binary classification; graded severity prediction remains for future work
\end{itemize}

\section{Final Remarks}

Depression remains a significant public health challenge, with diagnosis relying heavily on subjective self-report and clinical observation. Speech-based assessment offers the promise of objective, non-invasive markers that could enable earlier detection and more frequent monitoring.

This study demonstrates that such markers exist and can be identified using interpretable methods. The finding that spontaneous speech reveals richer depression signatures than reading tasks has immediate practical implications for system design. The identification of variability measures—particularly spectral dynamics, voice quality, and pausing—as key features provides clinically meaningful insights that could inform both automated systems and clinical practice.

Future work should validate these findings across diverse populations and languages, explore integration with clinical workflows, and investigate whether speech markers can track treatment response over time. The ultimate goal is a future where speech analysis contributes to early, accurate, and accessible mental health assessment.


%==============================================================================
% BIBLIOGRAPHY
%==============================================================================
\bibliographystyle{ieeetr}
\bibliography{references/dissertation}

%==============================================================================
% APPENDICES (optional)
%==============================================================================
\appendix

\chapter{Feature List}
The complete list of 88 eGeMAPS features used in this study is available in the project repository.

\chapter{Code Repository}
All code for this project is available at: \url{https://github.com/NileC03/dissertation}

The repository includes:
\begin{itemize}
    \item \texttt{scripts/extract\_features.py} — Feature extraction pipeline
    \item \texttt{scripts/run\_analysis.py} — Classification and importance analysis
    \item \texttt{scripts/advanced\_analysis.py} — Confusion matrices, significance tests, and visualisations
\end{itemize}

\end{document}
