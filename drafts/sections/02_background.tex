\chapter{Background}

This chapter provides the theoretical and technical foundation for the present work. Section~\ref{sec:depression} introduces depression as a clinical condition, its prevalence, and current diagnostic approaches. Section~\ref{sec:speech-production} explains how depression affects cognitive processes and, consequently, speech production. Section~\ref{sec:speech-features} formally defines the acoustic and prosodic features used in speech analysis. Section~\ref{sec:ml-approaches} reviews machine learning approaches to automatic depression detection. Finally, Section~\ref{sec:related-work} surveys key related work in the field.

\section{Depression: Clinical Overview}
\label{sec:depression}

\subsection{Definition and Prevalence}

Major Depressive Disorder (MDD), commonly referred to as depression, is a mental health condition characterised by persistent feelings of sadness, hopelessness, and loss of interest in activities~\cite{WHO2024}. According to the World Health Organization, depression affects over 280 million people globally, making it one of the leading causes of disability worldwide~\cite{WHO2024}.

In the United Kingdom, approximately 1 in 6 adults experiences depression or anxiety in any given week~\cite{McManus2016}. The economic burden is substantial: depression costs the UK economy an estimated £105 billion annually through healthcare costs, lost productivity, and reduced quality of life~\cite{OECD2018}.

\subsection{Current Diagnostic Methods}

Depression diagnosis traditionally relies on clinical interviews and standardised questionnaires. The two most widely used instruments are:

\begin{itemize}
    \item \textbf{Patient Health Questionnaire (PHQ-9/PHQ-8):} A self-report instrument where patients rate the frequency of nine depressive symptoms over the past two weeks on a 0--3 scale~\cite{Kroenke2001}. Scores of 10 or above indicate clinically significant depression.
    
    \item \textbf{Beck Depression Inventory (BDI):} A 21-item self-report questionnaire measuring characteristic attitudes and symptoms of depression~\cite{Beck1961}.
\end{itemize}

However, these methods have significant limitations:

\begin{enumerate}
    \item \textbf{Subjectivity:} Inter-rater reliability between clinicians can be as low as 0.28 kappa for some depression subtypes~\cite{Regier2013}.
    
    \item \textbf{Self-report bias:} Patients may underreport symptoms due to stigma or lack of insight into their condition.
    
    \item \textbf{Resource requirements:} Clinical assessments require trained professionals and are time-consuming, limiting scalability.
    
    \item \textbf{Access barriers:} Many individuals, particularly in underserved areas, lack access to mental health professionals.
\end{enumerate}

These limitations motivate the search for objective, scalable biomarkers of depression---among which speech has emerged as a promising candidate.


\section{Depression and Speech Production}
\label{sec:speech-production}

\subsection{The Cognition-Speech Connection}

Speech production is a complex cognitive process involving planning, motor control, and real-time monitoring~\cite{Levelt1989}. Depression affects multiple cognitive domains that directly influence speech:

\begin{itemize}
    \item \textbf{Psychomotor retardation:} Slowing of physical and mental processes, leading to reduced speech rate and longer pauses~\cite{Sobin1997}.
    
    \item \textbf{Reduced motivation:} Diminished drive and anhedonia affect prosodic expression, resulting in ``flat'' or monotonous speech~\cite{Cummins2015}.
    
    \item \textbf{Cognitive load:} Rumination and negative thought patterns consume cognitive resources, affecting fluency and articulation~\cite{Joormann2011}.
    
    \item \textbf{Emotional blunting:} Reduced emotional reactivity manifests as decreased pitch variation and vocal energy~\cite{Bylsma2008}.
\end{itemize}

Crucially, many of these changes are \textit{involuntary}---speakers cannot easily mask or fake them, making speech a potentially robust biomarker.

\subsection{Observable Speech Changes in Depression}

Research has identified several speech characteristics that differ between depressed and healthy individuals:

\begin{itemize}
    \item \textbf{Temporal features:} Slower speech rate, longer pause durations, increased hesitations~\cite{Cannizzaro2004}.
    
    \item \textbf{Prosodic features:} Reduced pitch range, lower mean fundamental frequency (F0), less pitch variability~\cite{Nilsonne1988}.
    
    \item \textbf{Voice quality:} Increased breathiness, jitter, and shimmer; reduced harmonic-to-noise ratio~\cite{Ozdas2004}.
    
    \item \textbf{Articulation:} Less precise consonants, reduced vowel space~\cite{Scherer2015}.
\end{itemize}


\section{Acoustic and Prosodic Features}
\label{sec:speech-features}

This section formally defines the speech features used in depression detection research. Features are typically extracted using signal processing toolkits such as OpenSMILE~\cite{Eyben2010} or Praat~\cite{Boersma2001}.

\subsection{Fundamental Frequency (F0)}

The fundamental frequency, denoted F0, corresponds to the rate of vocal fold vibration and is perceived as pitch. For a speech signal $x(t)$, F0 is the lowest frequency component of the periodic glottal waveform.

Common F0-derived features include:
\begin{itemize}
    \item Mean F0: $\bar{F0} = \frac{1}{N}\sum_{i=1}^{N} F0_i$
    \item F0 standard deviation: measures pitch variability
    \item F0 range: $F0_{max} - F0_{min}$
\end{itemize}

Depressed speakers typically exhibit lower mean F0 and reduced F0 variability~\cite{Cummins2015}.

\subsection{Energy and Intensity}

Signal energy, often measured as Root Mean Square Energy (RMSE), reflects vocal effort:

\begin{equation}
    RMSE = \sqrt{\frac{1}{N}\sum_{i=1}^{N} x_i^2}
\end{equation}

Depressed speakers often show reduced overall energy and less dynamic range~\cite{Scherer2015}.

\subsection{Mel-Frequency Cepstral Coefficients (MFCCs)}

MFCCs are spectral features that approximate human auditory perception. The extraction process involves:

\begin{enumerate}
    \item Compute the power spectrum via Short-Time Fourier Transform
    \item Apply mel-scale filterbank
    \item Take the logarithm of filterbank energies
    \item Apply Discrete Cosine Transform
\end{enumerate}

Typically, the first 12--13 MFCCs (excluding C0) are used, along with their first and second derivatives (delta and delta-delta coefficients). MFCCs capture spectral envelope characteristics related to articulation and voice quality.

\subsection{Temporal Features}

Key temporal features include:
\begin{itemize}
    \item \textbf{Speech rate:} Syllables or phonemes per second
    \item \textbf{Pause duration:} Length of silent intervals
    \item \textbf{Speech-to-pause ratio:} Proportion of voiced vs. unvoiced segments
    \item \textbf{Voice onset time:} Duration before voicing begins
\end{itemize}

\subsection{Voice Quality Features}

\begin{itemize}
    \item \textbf{Jitter:} Cycle-to-cycle variation in F0
    \item \textbf{Shimmer:} Cycle-to-cycle variation in amplitude
    \item \textbf{Harmonic-to-Noise Ratio (HNR):} Ratio of periodic to aperiodic energy
\end{itemize}

Higher jitter and shimmer, and lower HNR, indicate less stable phonation---patterns often observed in depression~\cite{Ozdas2004}.

\subsection{Standard Feature Sets}

To enable reproducibility, researchers have developed standardised feature sets:

\begin{itemize}
    \item \textbf{eGeMAPS:} The extended Geneva Minimalistic Acoustic Parameter Set comprises 88 features designed for affective computing~\cite{Eyben2016}.
    
    \item \textbf{ComParE:} The Computational Paralinguistics Challenge feature set includes over 6,000 features for comprehensive acoustic analysis~\cite{Schuller2013}.
\end{itemize}

This work employs features extracted using OpenSMILE, following the protocol established by Tao et al.~\cite{Tao2024} for the ANDROIDS corpus.


\section{Machine Learning Approaches}
\label{sec:ml-approaches}

\subsection{Traditional Machine Learning}

Early approaches to speech-based depression detection employed classical machine learning algorithms:

\begin{itemize}
    \item \textbf{Support Vector Machines (SVM):} Linear or kernel-based classifiers that find optimal separating hyperplanes. SVMs have been widely used due to their effectiveness with high-dimensional feature vectors~\cite{Cummins2015}.
    
    \item \textbf{Random Forests:} Ensemble methods combining multiple decision trees, offering robustness and built-in feature importance measures~\cite{Breiman2001}.
    
    \item \textbf{Logistic Regression:} Interpretable linear classifiers providing probability estimates and coefficient-based feature importance.
\end{itemize}

\subsection{Deep Learning Approaches}

Recent work has explored neural network architectures:

\begin{itemize}
    \item \textbf{Recurrent Neural Networks (RNNs):} Particularly Long Short-Term Memory (LSTM) networks, which can model temporal dependencies in speech sequences~\cite{Hochreiter1997}.
    
    \item \textbf{Convolutional Neural Networks (CNNs):} Applied to spectrogram representations of speech for automatic feature learning~\cite{Ma2016}.
    
    \item \textbf{Attention Mechanisms:} Methods that learn to weight the most relevant portions of input, such as the Multi-Local Attention approach proposed by Tao et al.~\cite{Tao2023MLA}.
\end{itemize}

\subsection{Evaluation Metrics}

Depression detection is typically framed as binary classification (depressed vs. control) or regression (predicting severity scores). Common metrics include:

\begin{itemize}
    \item \textbf{Accuracy:} Proportion of correct predictions
    \item \textbf{F1 Score:} Harmonic mean of precision and recall
    \item \textbf{Concordance Correlation Coefficient (CCC):} For regression tasks, measuring agreement between predicted and actual values
    \item \textbf{Area Under ROC Curve (AUC):} Discrimination ability across thresholds
\end{itemize}


\section{Related Work}
\label{sec:related-work}

\subsection{The AVEC Challenges}

The Audio/Visual Emotion Challenge (AVEC) series has been instrumental in advancing depression detection research. The 2016, 2017, and 2019 challenges used the DAIC-WOZ corpus~\cite{Gratch2014}, comprising clinical interviews with a virtual agent.

Key findings from AVEC include:
\begin{itemize}
    \item Multimodal approaches (audio + video + text) outperform unimodal methods
    \item Deep learning achieves state-of-the-art results but requires substantial data
    \item Baseline systems achieve approximately 5--6 MAE on PHQ-8 prediction
\end{itemize}

\subsection{The ANDROIDS Corpus}

Tao et al.~\cite{Tao2023ANDROIDS} introduced the ANDROIDS corpus, addressing several limitations of existing datasets:

\begin{itemize}
    \item \textbf{Professional diagnosis:} Labels assigned by psychiatrists, not self-report questionnaires
    \item \textbf{Dual speech tasks:} Both read speech (fairy tale) and spontaneous speech (interview)
    \item \textbf{Public availability:} Enables reproducibility and comparison
\end{itemize}

Using LSTM baselines, they achieved 83.4\% accuracy on read speech and 81.6\% on spontaneous speech---comparable to general practitioners' diagnostic accuracy.

\subsection{Feature Importance in Depression Detection}

While most research focuses on maximising classification accuracy, fewer studies examine which features drive predictions. Cummins et al.~\cite{Cummins2015} reviewed the literature and identified prosodic features (F0, speech rate) as consistently predictive. However, systematic feature importance analysis using modern interpretability methods (e.g., SHAP values) remains underexplored.

This gap motivates the present work, which aims to identify the most predictive acoustic features and compare their importance across speech tasks.


\section{Summary}

This chapter established the theoretical foundation for speech-based depression detection:

\begin{itemize}
    \item Depression is a prevalent condition with significant diagnostic challenges
    \item Speech offers a non-invasive, scalable biomarker due to the cognition-speech connection
    \item Acoustic features can be formally defined and extracted using standard tools
    \item Machine learning approaches range from classical SVMs to deep attention networks
    \item The ANDROIDS corpus provides an ideal testbed with professional labels and dual speech tasks
\end{itemize}

The following chapter describes the methodology employed to analyse feature importance in depression detection.
